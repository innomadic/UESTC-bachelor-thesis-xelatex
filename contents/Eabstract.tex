% !TEX TS-program = xelatex
% !TEX root = ../Thesis_Guo2013.tex
% !Mode:: "TeX:UTF-8"
\begin{Eabstract}{diverging moments}{asymptotics}{equiprobable partition}{power-law}{memory}
Moments, such as mean, variance and covariance, are important statistics that characterize the shape of distributions. Every moment simultaneously corresponds to a sample moment and a population moment. A specific population moment may diverge for heavy-tailed distributions. When population moments do not diverge, sample moments will converge to their corresponding population moments by probability. However, when population moments diverge, sample moments also keep growing with the sample size. Firstly in this thesis, the \textit{equiprobable partition method} (EPM) is presented for systematically analyzing the behavior of diverging sample moments. Estimators for sample moments are constructed from EPM, which are then used to theoretically derive the asymptotics for sample moments when they diverge. EPM estimators are shown to be unbiased under convergence and their accuracy is also numerically studied. Finally, we study the autocorrelation of power-law series, where EPM is applied to the analysis of diverging moments involved. We find non-trivial bounds for the first-order autocorrelation as a function of the power-law exponent, which agree with numerical experiments and empirical data.
\end{Eabstract}
\setcounter{page}{2}
