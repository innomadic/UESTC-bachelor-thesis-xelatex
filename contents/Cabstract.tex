% !TEX TS-program = xelatex
% !TEX root = ../Thesis_Guo2013.tex
% !Mode:: "TeX:UTF-8"

\begin{Cabstract}{发散矩}{渐近}{等概率分割}{幂律}{记忆性}
矩,诸如均值、方差、协方差,是刻画概率分布形态的重要统计量。每个矩同时对应于一个样本矩和一个总体矩。当总体矩收敛时,随着样本容量的增大,样本矩依概率收敛于总体矩;然而,对于重尾分布,总体矩可能发散,此时样本矩会随着样本容量的增加而不断增大。本文提出等概率分割方法(EPM)用于系统地分析发散情形下有限个样本的矩随样本容量增大的渐近行为。本文证明了EPM在收敛条件下的准确性,并对其给出的发散条件下的渐近线进行了数值分析。最后,针对幂律分布时间序列的自相关,运用EPM处理其中的发散矩项,从而得出其一阶自相关的非平凡上下界,该上下界与数值模拟和实证数据相吻合。
\end{Cabstract}